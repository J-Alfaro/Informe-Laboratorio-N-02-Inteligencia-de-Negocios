\begin{center}
    PRACTICA DE LABORATORIO N° 02
\end{center}

\section{OBJETIVOS}
Crear relaciones automáticas y manuales

\section{REQUERIMIENTOS}

\begin{itemize}

- Conocimientos básicos de administración de base de datos Microsoft   SQL Server.
\\- Conocimientos básicos de SQL.
\\- Microsoft SQL Server 2016 o superior
\\- Base de datos AdventureWorks2016 o superior
\\- Power BI Desktop.
\\- Tener una cuenta Microsoft registrada en el Portal de Power Bi.
\end{itemize}

\section{DESARROLLO} 

\begin{itemize}
1. Ingresar a Power BI Desktop, en el cuadro de dialogo Obtener Datos (Get Data), asegurarse que Excel esta seleccionado y hacer click en Conectar (Connect), buscar el archivo Adventure Works Sales Data.xlsx y hacer click en Cargar (Load).

\end{itemize} 

\begin{center}
\includegraphics[width=15cm]{./Imagenes/img1} 
\end{center}
\newpage
\begin{itemize}
2. En el cuadro de Administrar relaciones (Manage Relationships) realizar las configuraciones indicadas 
\end{itemize} 

\begin{center}
\includegraphics[width=15cm]{./Imagenes/img2} 
\end{center}

\begin{itemize}
3. Abrir el archivo Adventure Works Product Categories.xlsx
\end{itemize} 

\begin{center}
\includegraphics[width=15cm]{./Imagenes/img4} 
\end{center}


\newpage
\begin{itemize}
4. En el panel Campos, haga clic en DimCustomer,en la cinta Modelado, en el grupo Cálculos, haga clic en Nueva columna.En la barra de fórmulas, resalte Columna = y escriba:
IncomeStatus = IF (DimCustomer[YearlyIncome] < 25000, "Lower Income",
IF (AND(DimCustomer[YearlyIncome] >= 25000, DimCustomer[YearlyIncome] < 60000),
"Middle Income",
IF (AND(DimCustomer[YearlyIncome] >= 60000, DimCustomer[YearlyIncome] < 100000),
"Higher Income",
IF (DimCustomer[YearlyIncome] >= 100000, "Very High Income", "Other")))) y finalmente presione Enter.
\end{itemize}

\begin{center}
\includegraphics[width=15cm]{./Imagenes/img6} 
\end{center}



\begin{itemize}
5. En el panel Campos, haga clic en FactInternetSales.En la cinta Modelado, en el grupo Cálculos, haga clic en Nueva columna y realizar la siguiente operacion:
Profit = CURRENCY(FactInternetSales[UnitPrice] -
FactInternetSales[ProductStandardCost])
\end{itemize}

\begin{center}
\includegraphics[width=15cm]{./Imagenes/img8} 
\end{center}